\documentclass[12pt]{article}
\usepackage{expex}


\begin{document}

Tink'amwaj re jun saqrab'eb' ri. Chwaj tinye' jb'ijk nen jb'antaj {Tz'unun Kaab'}. Xkaj yole' taq qamaam, qatiit. \'Ojr taq kristyan neri, xijyol taq. {Nen chak} k'uli? Nen xtamaj b'a Xoqoneb'? Tjyol taq qamaam, qatiit' wi' jun qub' ajxojob', t'el taq l nimq'iij qatinmit. Y taw nojel ab', tijb'ij taq, Tqatok qapach, tqamol qib', tqesaj x\'ojol. Xtaw taq chuch. Xtok taq jpachaq. Xnuk' jyoljaq. Y wi' juun xij, "Inel ak'laq. Xtaw q'iij tijmet tijme taqch ritz'yaq ajx\'ojol li K'\'iche' o rik'il Ajm\'axb'." Y wi' jun rechaq xmay jwiich, "Nin tinb'an?" Ta' inpuwaq. Ya ttaw q'iij oj'el l x\'ojol. Y nin tinb'an ri? Tb'ison r\'anm. Ta' tta'w nen tran. Xk'am b'i r\'ikaj. Xk'am b'i jk\'olb'. Xk'am b'i jp\'atn. Xe' b'a Xoqoneb', tb'isonk. Ta' {nen taj} nab'ej, ri tril, xelch jun nimlaj wunaq b'a k'achee'laj y xij re, "Nen tab'an neri?" Sii' tintok. Per nen tamay? Ta' nen tinmay. Per iin tinwil que wi' jun b'is laj aw\'anm. B'ij chwe, atint'o'w. Per nin inat'o'w si aat, ta' awetam lo que wi' laj w\'anm, wetami'n. Siw tab'ij chwe, atint'o'w. Tb'ison r\'anm, ta' raj tb'ij re. Per iin chwaj inat'o'w, per aat ta' atkwiink. Inkwiini'n! B'ij nika chwe nen b'is wi' laj aw\'anm. Ya ttaw jq'iij qatinmit y wi'n l xjooj. Ya inpach, ya tb'ijme taqch ritz'yaqaq y iin ta' inpuwaq. Ahhh, ri li tamay? Tche re, "Ji'n." Ya ttaw hora atxojoow taq. Ya ttaw hor re nimq'ij. Ri li tamay? Iin atint'o'w! Per nin inat'o'w re? Ta' tab'ij y atint'o'w. Katnatun b'i jli y tinb'ij chawe nen tinya' chawe. "Tzib' kali," tche. Xnatunch ra jxukt tinmit. Y cuand xq'aj chrij, tb'ij re, "Ri li itz'yaq chawaj?" Ri li lo que tamay? Cha' jun awe y tak'am b'ik per ni jumul tab'ij re kristyan lamas xame'w. Pax r\'anm tril ritz'yaq ajx\'ojol, tqopqon, tqopqon, l puwaq. Y tijb'ij, "{Nen k'uli} xinb'an ri, ri li xyuqinte' y ri li tink'am b'ik?" Va cha' jun awe y tak'am b'ik y ri takoj laj mer jq'\'{\i}jl qatinmit. Per tinb'ij b'i chawe ta' tab'ij re ni jun, que neri xyuqame'w. "Jaan," xche'. Xk'amch atz'yaq, tki'kot r\'anm. Ta' chki xk'amch jsii'. Ta' chki xk'amch ni'j qleen. Reqajch atz'yaq, xk'amch chirchooch, y xyuq chirchooch ta' xij re rix\'oql, ni jun. Y jpach t'el taq l xjooj xij taq re: "Y ak\'uchj? Nin tab'an re?" "No, iin ta' tinye' ink\'uchj." Per nin tab'an re awitz'yaq? Pues nin tinb'an tre. Tinwilb'ej na titaw q'iij y {rik'u re'} tiinjutun chaxo'laq. Jaan xche' taq re, "Per {nin k'u} tran wunaq li? Ta' ta' xya' jk\'uchj re ritz'yaq." "Y nin tqab'an re?" "Wa'x naq kla', tril li jcholajl." Xtaw jq'\'ijl roox q'iij re jnimq'iij Simyel. Xjoow taq y xkojch ritz'yaq wunaq, Tqopqon! Tqopqon!, l q'aaq' ritz'iiyaq. Y xril taq juntir. "Lamas xame'w awitz'yaq?" "Ta' ke' apuwaq, ta' ameer, ta' akortiil." "At b'eyom ta'n, y {nin k'u} man atz'yaq, li li xame'w?" Ta' tjb'ij. Y ta' tb'ij, tjk'ot taq jchii', "Per lamas xame'w awitz'yaq?" Ta' tb'ij, va, taxan jun wi' jun chijxo'laq tb'ij re: "qaye' awryent re, qaq'ab'rsaj." "Ya sache'l chaq jb'aa, tjyola' li chqe lamas xijme'w." "Jaan," xche' taq, xkoj, xya' taq k\'umb'l re. Xq'ab'rsaj taq. Ya laj q'ab'ark, xelch laj chii'. Tijb'ij rechaq, "Pues lamastch xinme' witz'yaq, jli b'a loom." "Lamas loom?" "Jli b'a Xoqoneb'." "{Nin k'u} xab'an?" "Xelch jun nimlaj wunaq chwiij." "Y xij chwe que {nen chak} inb'esonk?" Cuando wi' laj q'aab' re re utziil re jt'o'w y kla' xinme'ch witz'yaq. K'uttaj chq\'awch! Loq'ori ta' chki inkwiin, ta' chki tinte'. Ta' chki tinte'. Y kla' k'uli tjyol taq qamaam qatiit'. Que kla' wi' jun meer jkortiil qatinmit {Tz'unun Kaab'}). {Nin k'u chak} xq'aje b'ik? Nin k'u chak xsach b'ik? Mat xijyol na che'. Mat xel na laj chii' wunaq, xijk'ol taj. Xijch'uqt jb'aa lo que xijsaj re, jpuwaq qatinmit Tz'unun Kaab', {ajwi' tna} re. Cuand xretmaj wunaq, retmaj nimlaj wunaq. Que fue relatado la pobreza de nuestro pueblo, ese mismo d\'ia que lo relat\'o. Xtormaj jun lel B'a Kameb', xnatun b'ik K\'ob'n. Xnatun b'ik Rab'inal, kla'. Xtok b'i jb'ee jmeer jkortiil. J'upuwaq Tz'unun Kaab'. K'ixk'olil xtamaj \'ojr taq tziij. \'Ojr taq qamaam qatiit' tb'eson taq. Tajb'ij taq, kla' tawem \'ojr, nimlaj k'achee'laj. Ta' qas ttaw jun kla' porque k'ani'n, k'ani'n loq'laj mundo kla'. Ti li xkan yolsaaj jwi'l qaqaaj Jos\'e M\'endez. Ri li qamaam qatiit' xkan yoloow taq y xi'j taq, "Xqak'ol tne." Xqachajaj tne jloq'\'oxl y loq'b'al puwaq re qatinmit Tz'unun Kaab'. Ri li tinyol jun saqrab'em ri rej. Rik'al junab' y tink'amwaj chawe ataq. 

 Agradezco por esta ma\~nana. Quiero dar a conocer cu\'al es la cultura de Uspant\'an. Dejaron relatado nuestros abuelos. Antiguas personas de aqu\'i, relataron. ?`por qu\'e? ?`Qu\'e pas\'o en Xoqoneb'? Cuentan nuestros abuelos, que hab\'ia un grupo del baile de la conquista, que sal\'ian en la fiesta de nuestro pueblo. Y al llegar todos los a\~nos, decian, "Busquemos nuestros compa\~neros, reunamonos, saquemos el baile de la conquista." Llegaron a la fecha. Buscaron a sus compa\~neros. Se pusieron de acuerdo. Y hubo uno que dijo: "Salgo con ustedes. Lleg\'o el d\'ia para que fueran a traer los trajes del baile de la conquista a Quich\'e o con los chichicastecos." Y uno de ellos se preocup\'o, "?`Qu\'e voy a hacer?" No tengo dinero. Ya casi llega el d\'ia para que salgamos en el baile de la conquista. ?`Y qu\'e voy a hacer? Estaba triste. No encontraba que hacer. Se llev\'o su acha. Se llev\'o su lazo. Se llev\'o su mecapal. Se fue a Xoqoneb', triste. De repente, vio, que sali\'o un hombre grande sobre la monta\~na y le pregunt\'o, "?`Qu\'e haces aqu\'i?" Le\~na estoy buscando. ?`Pero qu\'e te preocupa? No me preocupa nada. Pero yo veo que hay una tristeza en tu coraz\'on. Decimelo, te voy a ayudar. Pero en qu\'e me vas a ayudar si no sabes lo que tengo en mi coraz\'on, lo s\'e. Si me decis, te voy a ayudar. Muy triste, no quer\'ia decirle. Pero yo quiero que me ayudes, pero no podes. !`Bien puedo! Decime qu\'e tristeza hay en tu coraz\'on. Ya llega la fiesta en nuestro pueblo y estoy en el baile. Ya mis compa\~neros, ya van a ir a traer sus trajes y yo no tengo dinero. ?`Ahhh, eso es lo qu\'e te preocupa? Le dijo, "S\'i eso." Ya llega la hora para que bailen. Ya llega la hora para la fiesta. ?`Eso te preocupa? !`Yo te voy a ayudar! ?`Pero en qu\'e me vas a ayudar? No digas nada y te voy a ayudar. Volteate para all\'a y te voy a decir que te voy a dar. "Esta b\'ien," dijo. Se volte\'o viendo hacia a un lado del pueblo. Y cuando se volte\'o, le pregunt\'o, "?`Esos trajes queres?" ?`Esto es lo que te preocupa? Escoge un tuyo y te lo llevas, pero no tenes que decirles nada a las personas, donde lo fuiste a traer. Se quebrant\'o su coraz\'on al ver los trajes del baile de la conquista que brillaban, brillaban entre el oro. Y se preguntaba, "?`Qu\'e habr\'e hecho para venir a encontrar esto y es lo que me voy a llevar?" Va escoge un tuyo y te lo llevas y ese vas a usar en el mero d\'ia de fiesta de nuestro pueblo. Pero de una vez te lo digo no le digas a nadie, que aqu\'i lo veniste a traer. "Bueno," dijo. Se trajo el vestuario, muy feliz. Ya no se trajo su le\~na. Ya no se trajo ninguna cosa. Se carg\'o el traje, se los trajo para su casa y vino a su casa no le dijo nada a su esposa, ni a sus familiares, a nadie. Y sus compa\~neros del baile, le preguntaron: "?`Y tu cuota? ?`Qu\'e vas a hacer?" "No, yo no voy a dar mi cuota." ?`Pero qu\'e vas a hacer con tu traje? Pues qu\'e puedo hacer. Tengo que esperar que llegue el d\'ia y despu\'es me meto entre ustedes. Bueno le dijeron, "?`Pero qu\'e va ha hacer ese hombre? No dio su cuota para su traje." "?`Y qu\'e le vamos a hacer?" "Que se quede ah\'i, que vea que hacer." Lleg\'o la festividad el tercer d\'ia de la fiesta de San Miguel. Bailaron y se puso su traje el hombre, !`Brillaba! !`Brillaba!, en la luz su traje. Y lo vieron todos. "?`D\'onde fuiste a traer tu traje?" "No ten\'ias dinero, no tenes riquezas." "No ten\'ias nada y ese traje ?`d\'onde lo fuiste a traer?" No dec\'ia nada. Y no lo dec\'ia, lo interrogaban, "?`Pero d\'onde fuiste a traer tu traje?" No lo dec\'ia, va, vino uno que estaba entre ellos dijo: "Demosle licor, emborrachemoslo." "Ya inconsciente, nos ir\'a a contar donde lo fue a traer." "Bueno," dijeron, le dieron licor. Lo emborracharon. Ya entre su borrachera, sali\'o de su boca. Y les dijo, "Pues donde fui a traer mi traje, all\'a sobre el cerro." "?`Qu\'e cerro?" "All\'a en Xoqoneb'." "?`Qu\'e hiciste?" "Sali\'o un hombre grande detr\'as de mi." "Y me dijo que ?`por qu\'e estas triste? Cuando ten\'ia entre sus manos alegr\'ia para ayudar y ah\'i fui a traer mi traje. !`Anda a ense\~narnos! Ahorita ya no puedo, ya no lo encuentro. Ya no lo encuentro. Y es as\'i como cuentan nuestros abuelos. Que ah\'i estaba la riqueza de nuestro pueblo Uspant\'an. ?`Por qu\'e se regreso? ?`Por qu\'e se perdio? No lo hubiera contado dice. No hubiera salido de la boca del hombre, lo hubiera guardado. No hubiera olvidado lo que le dijeron, el dinero de nuestro pueblo de Uspant\'an, a\'un estuviera. Cuando se enter\'o el hombre, se enter\'o el hombre grande. Que fue relatado la pobreza de nuestro pueblo, ese mismo d\'ia que lo relat\'o. Se abri\'o una ventana en B'a Kameb', viendo hacia Cob\'an. Vio hacia Rabinal, ah\'i. Busc\'o su camino la riqueza. El dinero de Uspant\'an. Dificultades se dieron en la antiguedad. Nuestros abuelos quedaron tristes. Dec\'ian, que al llegar ah\'i antes, era una gran monta\~na. No llegaba uno ah\'i porque era enojado, enojado el sagrado mundo ah\'i. Eso fue relatado por Don Jos\'e M\'endez. Fueron nuestros abuelos quienes lo relataron y dijeron, "Lo hubieramos guardado." Hubieramos cuidado el sagrado dinero de nuestro pueblo Uspant\'an. Esto es lo que yo relato en esta ma\~nana. En este a\~no y les agradezco a ustedes.

\ex
\begingl
  \gla  T-\O-in-k'amwaj re jun saqrab'eb' ri. //
  \glb inc-a3-e1s-agradecer dem uno ma\~nana dem //
  \glft `Agradezco por esta ma\~nana.' //
\endgl
\xe

\ex
\begingl
  \gla  Ch-\O-w-aj t-\O-in-ye' j-b'ij-k nen j-b'an-taj \{Tz'unun Kaab'\}. //
  \glb inc-a3-e1s-querer inc-a3-E1s-give e3-decir-sv int e3-do-pas Uspant\'an //
  \glft `Quiero dar a conocer cu\'al es la cultura de Uspant\'an. //
\endgl
\xe

\ex
\begingl
  \gla  X-\O-\O-kaj yol-e' taq qa-maam, qa-tiit.' //
  \glb com-a3-e3-dejar decir-sc pl e1p-abuelo e1p-abuela //
  \glft `Dejaron relatado nuestros abuelos.' //
\endgl
\xe

\ex
\begingl
  \gla \'Ojr taq kristyan neri, x-\O-ij-yol taq. //
  \glb  antiguamente pl persona aqu\'i com-a3-e3-relatar pl //
  \glft `Antiguas personas de aqu\'i, relataron.' //
\endgl
\xe


\ex
\begingl
  \gla  \{Nen chak\} k'uli? //
  \glb int part //
  \glft `?`por qu\'e?' //
\endgl
\xe

\ex
\begingl
  \gla  Nen x-\O-tamaj b'a Xoqoneb'? //
  \glb int com-a3-encontrar pre Xoqoneb' //
  \glft `?`Qu\'e pas\'o en Xoqoneb'?' //
\endgl
\xe

\ex
\begingl
  \gla  T-\O-j-yol taq qa-maam, qa-tiit' wi' jun qub' aj-xoj-ob', t-\O-'el taq l nimq'iij qa-tinmit. //
  \glb  inc-a3-e3-contar pl e1p-abuelo e1p-abuela exs uno grupo agt-bailar-pl inc-a3-salir pl pre fiesta e1p-pueblo //
  \glft `Cuentan nuestros abuelos, que hab\'ia un grupo del baile de la conquista, que sal\'ian en la fiesta de nuestro pueblo.' //
\endgl
\xe

\ex
\begingl
  \gla  Y taw nojel ab', t-\O-ij-b'ij taq, T-\O-qa-tok qa-pach, t-\O-qa-mol q-ib', t-\O-q-esaj x\'ojol. //
  \glb  y llegar todo a\~no inc-a3-e3-decir pl inc-a3-e1p-buscar e1p-compa\~nero inc-a3-e1p-reunir e1p-refl inc-a3-e1p-sacar baile //
  \glft `Y al llegar todos los a\~nos, decian, "Busquemos nuestros compa\~neros, reunamonos, saquemos el baile de la conquista."' //
\endgl
\xe


\ex
\begingl
  \gla  X-\O-taw taq chuch. //
  \glb com-b3-llegar pl fecha //
  \glft `Llegaron a la fecha.' //
\endgl
\xe

\ex
\begingl
  \gla  X-\O-tok taq j-pach-aq. //
  \glb com-a3-buscar pl e3-compa\~nero-pl //
  \glft `Buscaron a sus compa\~neros.' //
\endgl
\xe

\ex
\begingl
  \gla  X-\O-nuk' j-yolj-aq. //
  \glb  com-a3-juntar e3-habla-pl //
  \glft `Se pusieron de acuerdo.' //
\endgl
\xe

\ex
\begingl
  \gla Y wi' juun x-\O-\O-ij, "In-el a-k'l-aq. X-\O-taw q'iij t-ij-met t-\O-ij-me taq-ch r-itz'yaq aj-x\'ojol li K'\'iche' o r-ik'il Ajm\'ax-b'." //
  \glb and exs uno com-a3-e3-decir e1s-salir e2s-sr-pl com-e3-llegar d\'ia inc-a3-e3-recibir inc-a3-e3-recibir pl-dir e3-traje agt-baile pre Quiche o e3-sr Chichicasteco-pl //
  \glft `Y hubo uno que dijo: "Salgo con ustedes. Lleg\'o el d\'ia para que fueran a traer los trajes del baile de la conquista a Quich\'e o con los chichicastecos."' //
\endgl
\xe

\ex
\begingl
  \gla  Y wi' jun rechaq x-\O-\O-may j-wiich, "Nin t-\O-in-b'an?" //
  \glb y exs uno ellos com-a3-e3-penar e3-rostro int inc-a3-e1s-hacer //
  \glft `Y uno de ellos se preocup\'o, "?`Qu\'e voy a hacer?"' //
\endgl
\xe


\ex
\begingl
  \gla Ta' in-puwaq. //
  \glb neg e1s-dinero //
  \glft `No tengo dinero.' //
\endgl
\xe

\ex
\begingl
  \gla Ya t-\O-taw q'iij oj-'el li x\'ojol, //
  \glb ya inc-b3-llegar d\'ia a1p-salir pre baile //
  \glft `Ya casi llega el d\'ia para que salgamos en el baile de la conquista.' //
\endgl
\xe

\ex
\begingl
  \gla  Y nin t-\O-in-b'an ri? //
  \glb  y int inc-a3-e1s-hacer esto //
  \glft `?`Y qu\'e voy a hacer?' //
\endgl
\xe

\ex
\begingl
  \gla  T-\O-b'ison r-\'anm. //
  \glb inc-a3-estar.triste e3-coraz\'on //
  \glft `Estaba triste.' //
\endgl
\xe

\ex
\begingl
  \gla  Ta' t-\O-\O-ta'-w nen t-\O-r-an. //
  \glb  neg inc-a3-e3-encontrar-sc int inc-a3-e3-hacer //
  \glft `No encontraba que hacer.' //
\endgl
\xe

\ex
\begingl
  \gla  X-\O-k'am b'i r-\'ikaj. //
  \glb com-a3-traer dir e3-hacha //
  \glft `Se llev\'o su acha.' //
\endgl
\xe

\ex
\begingl
  \gla  X-\O-k'am b'i j-k\'olb'. //
  \glb  com-a3-traer dir e3-lazo //
  \glft `Se llev\'o su lazo.' //
\endgl
\xe

\ex
\begingl
  \gla  X-\O-k'am b'i j-p\'atn. //
  \glb com-a3-traer dir e3-mecapal //
  \glft `Se llev\'o su mecapal.' //
\endgl
\xe

\ex
\begingl
  \gla  X-\O-e' b'a Xoqoneb', t-\O-b'ison-k. //
  \glb com-a3-ir pre Xoqoneb' inc-a3-estar.triste-sc //
  \glft `Se fue a Xoqoneb', triste.' //
\endgl
\xe

\ex
\begingl
  \gla  Ta' \{nen taj\} nab'ej, ri t-\O-r-il, x-el-ch jun nim-laj wunaq b'a k'achee'laj y x-\O-ij re, "Nen t-\O-a-b'an neri?" //
  \glb  neg int primero dem inc-a3-e3-ver com-salir-dir uno grande-sup hombre pre selva y com-a3-decir dem int inc-a3-e2s-hacer aqu\'i //
  \glft `De repente, vio, que sali\'o un hombre grande sobre la monta\~na y le pregut\'o, "?`Qu\'e haces aqu\'i?"' //
\endgl
\xe

\ex
\begingl
  \gla  Sii' t-\O-in-tok. //
  \glb le~na inc-A3-e1s-buscar //
  \glft `Le\~na estoy buscando.' //
\endgl
\xe

\ex
\begingl
  \gla  Per nen t-\O-a-may? //
  \glb pero int inc-a3-e2s-penar //
  \glft `?`Pero qu\'e te preocupa?' //
\endgl
\xe

\ex
\begingl
  \gla  Ta' nen t-\O-in-may. //
  \glb neg int inc-a3-e1s-penar //
  \glft `No me preocupa nada.' //
\endgl
\xe


\ex
\begingl
  \gla Per iin t-\O-inw-il que wi' jun b'is laj aw-\'anm. //
  \glb pero yo inc-a3-e1s-see que exs uno tristeza pre e2s-coraz\'on //
  \glft `Pero yo veo que hay una tristeza en tu coraz\'on.' //
\endgl
\xe

\ex
\begingl
  \gla  B'ij ch-w-e, at-in-t'o'-w. //
  \glb decir pre-e1s-sr a2s-e1s-ayudar-sc //
  \glft `Decimelo, te voy a ayudar.' //
\endgl
\xe

\ex
\begingl
  \gla  Per nin in-a-t'o'-w si aat, ta' aw-etam lo que wi' laj w-\'anm, w-etam-i'n. //
  \glb pero int a1s-e2s-ayudar-sc si t\'u neg e2s-saber lo que exs pre e1s-coraz\'on e1s-coraz\'on-enf //
  \glft `Pero en qu\'e me vas a ayudar si no sabes lo que tengo en mi coraz\'on, lo s\'e.' //
\endgl
\xe

\ex
\begingl
  \gla  Si-w t-\O-a-b'ij ch-w-e, at-in-t'o'-w. //
  \glb  si-enf inc-a3-e2s-decir pre-e1s-sr a2s-e1s-ayudar-sc //
  \glft `Si me decis, te voy a ayudar.' //
\endgl
\xe

\ex
\begingl
  \gla  T-\O-b'ison r-\'anm, ta' raj t-\O-b'ij re. //
  \glb inc-a3-estar.triste e3-coraz\'on neg adv inc-a3-decir dem //
  \glft `Muy triste, no quer\'ia decirle.' //
\endgl
\xe

\ex
\begingl
  \gla Per iin ch-\O-w-aj in-a-t'o'-w, per aat ta' at-kwiin-k. //
  \glb pero yo inc-a3-e1s-querer a1s-e2s-ayudar-sc pero tu neg a2s-poder-sc //
  \glft `Pero yo quiero que me ayudes, pero no podes.' //
\endgl
\xe

\ex
\begingl
  \gla In-kwiin-i'n! B'ij nika ch-w-e nen b'is wi' laj aw-\'anm. //
  \glb a1s-poder-enf decir part pre-e1s-sr int tristeza exs pre e2s-coraz\'on //
  \glft `!`Bien puedo! Decime qu\'e tristeza hay en tu coraz\'on.' //
\endgl
\xe

\ex
\begingl
  \gla  Ya t-\O-taw j-q'iij qa-tinmit y wi'-n l xjooj. //
  \glb ya inc-b3-llegar  e3-d\'ia e1p-pueblo y exs-a1s pre baile //
  \glft `Ya llega la fiesta en nuestro pueblo y estoy en el baile.' //
\endgl
\xe

\ex
\begingl
  \gla  Ya in-pach, ya t-\O-b'i-j-me taq-ch r-itz'yaq-aq y iin ta' in-puwaq. //
  \glb ya a1s-compa\~nero ya inc-a3-dir-e3-recibir pl-dir e3-traje-pl y yo neg e1s-dinero //
  \glft `Ya mis compa\~neros, ya van a ir a traer sus trajes y yo no tengo dinero.' //
\endgl
\xe

\ex
\begingl
  \gla  Ahhh, ri li t-\O-a-may? //
  \glb ahhh dem pre inc-a3-e2s-penar //
  \glft `?`Ahhh, eso es lo qu\'e te preocupa?' //
\endgl
\xe

\ex
\begingl
  \gla  T-\O-che re, "Ji'-n." //
  \glb inc-a3-decir dem si-enf //
  \glft `Le dijo, "S\'i eso."' //
\endgl
\xe

\ex
\begingl
  \gla  Ya t-\O-taw hora at-xojoo-w taq. //
  \glb ya inc-b3-llegar hora a2s-bailar-sc pl //
  \glft `Ya llega la hora para que bailen.' //
\endgl
\xe

\ex
\begingl
  \gla  Ya t-\O-taw hor r-e nimq'ij. //
  \glb ya inc-b3-llegar hora e3-sr fiesta //
  \glft `Ya llega la hora para la fiesta.' //
\endgl
\xe

\ex
\begingl
  \gla  Ri li t-\O-a-may? //
  \glb dem pre inc-a3-e2s-penar //
  \glft `?`Eso te preocupa?' //
\endgl
\xe

\ex
\begingl
  \gla Iin at-in-t'o'-w! //
  \glb yo a2s-e1s-ayudar-sc //
  \glft `!`Yo te voy a ayudar!' //
\endgl
\xe

\ex
\begingl
  \gla Per nin in-a-t'o'-w re? //
  \glb pero int a1s-e2s-ayudar-sc dem //
  \glft `?`Pero en qu\'e me vas a ayudar?' //
\endgl
\xe

\ex
\begingl
  \gla  Ta' t-\O-a-b'ij y at-in-t'o'-w. //
  \glb neg inc-a3-e2s-decir y a2s-e1s-ayudar-sc //
  \glft `No digas nada y te voy a ayudar.' //
\endgl
\xe

\ex
\begingl
  \gla K-at-natun b'i jli y t-\O-in-b'ij ch-aw-e nen t-\O-in-ya' ch-aw-e. //
  \glb inc-a2-voltear dir all\'a and inc-a3-e1s-decir pre-a2s-sr int inc-a3-e1s-dar pre-e2s-sr //
  \glft `Volteate para all\'a y te voy a decir que te voy a dar.' //
\endgl
\xe

\ex
\begingl
  \gla  "Tzib' kali," t-\O-che. //
  \glb word calidad inc-a3-decir //
  \glft `"Esta b\'ien," dijo.' //
\endgl
\xe

\ex
\begingl
  \gla  X-\O-natun-ch ra j-xukt tinmit. //
  \glb com-a2-voltear-dir dem e3-sr pueblo //
  \glft `Se volte\'o viendo hacia a un lado del pueblo.' //
\endgl
\xe

\ex
\begingl
  \gla  Y cuand x-\O-q'aj ch-r-ij, t-\O-b'ij re, "Ri li itz'yaq ch-\O-a-waj?" //
  \glb  y cuand com-a3-regresar pre-e3-sr inc-a3-decir dem dem pre traje inc-a3-e2s-querer //
  \glft `Y cuando se volte\'o, le pregunt\'o, "?`Esos trajes queres?"' //
\endgl
\xe

\ex
\begingl
  \gla  Ri li lo que t-\O-a-may? //
  \glb dem pre los que inc-a3-e2s-penar //
  \glft `?`Esto es lo que te preocupa?' //
\endgl
\xe

\ex
\begingl
  \gla Cha' jun aw-e y t-\O-a-k'am b'i-k per ni jumul t-\O-a-b'ij r-e kristyan lamas x-\O-a-me'-w. //
  \glb escoger uno e2s-sr and inc-a3-e2s-traer dir-sc pero ni siempre inc-a3-e2s-decir e3-sr persona int inc-a3-e2s-recibir-enf //
  \glft `Escoge un tuyo y te lo llevas, pero no tenes que decirles nada a las personas, donde lo fuiste a traer.' //
\endgl
\xe

\ex
\begingl 
  \gla Pax r-\'anm t-\O-r-il r-itz'yaq aj-x\'ojol, t-\O-qop-qo-n, t-\O-qop-qo-n, l puwaq. //
  \glb quebrar e3-coraz\'on inc-a3-e3-ver e3-traje agt-baile inc-a3-brillar-red-ap inc-a3-brillar-red-ap pre dinero //
  \glft `Se quebrant\'o su coraz\'on al ver los trajes del baile de la conquista que brillaban, brillaban entre el oro.' //
\endgl
\xe

\ex
\begingl
  \gla Y t-\O-ij-b'ij, "Nen k'uli x-\O-in-b'an ri, ri li x-\O-yuq-in-te' y ri li t-\O-in-k'am b'i-k?" //
  \glb and inc-a3-e3-decir int part com-a3-e1s-hacer dem dem pre com-a3-dir-e1s-encontrar and dem pre inc-a3-e1s-traer dir-sc //
  \glft `Y se preguntaba, "?`Qu\'e habr\'e hecho para venir a encontrar esto y es lo que me voy a llevar?"' //
\endgl
\xe

\ex
\begingl
  \gla Va cha' jun aw-e y t-\O-a-k'am b'i-k y ri t-\O-a-koj laj mer j-q'\'\{\i\}ij-l qa-tinmit. //
  \glb vaya escoger uno e2s-sr and inc-a3-e2s-traer dir-sc and dem inc-a3-e2s-usar pre mero e3-d\'ia-pre e1p-pueblo //
  \glft `Va escoge un tuyo y te lo llevas y ese vas a usar en el mero d\'ia de fiesta de nuestro pueblo.' //
\endgl
\xe

\ex
\begingl
  \gla  Per t-\O-in-b'ij b'i ch-aw-e ta' t-\O-a-b'ij r-e ni jun, que neri x-\O-yuq-a-me'-w. //
  \glb pero inc-a3-e1s-decir dir pre-a2s-sr neg inc-a3-e2s-decir e3-sr ni uno que adv com-a3-dir-a2s-recibir-sc //
  \glft `Pero de una vez te lo digo no le digas a nadie, que aqu\'i lo veniste a traer.' //
\endgl
\xe

\ex
\begingl
  \gla  "Jaan," x-\O-che'. //
  \glb  bueno com-a3-decir //
  \glft `"Bueno," dijo.' //
\endgl
\xe

\ex
\begingl
  \gla  X-\O-\O-k'am-ch atz'yaq, t-\O-ki'kot r-\'anm. //
  \glb  com-e3-A3-traer-dir traje inc-A3-estar.feliz e3-coraz\'on //
  \glft `Se trajo el vestuario, muy feliz.' //
\endgl
\xe

\ex
\begingl
  \gla  Ta' chki x-\O-\O-k'am-ch j-sii'. //
  \glb neg neg com-e3-a3-traer-dir e3-le\~na //
  \glft `Ya no se trajo su le\~na.' //
\endgl
\xe

\ex
\begingl
  \gla  Ta' chki x-\O-\O-k'am-ch ni'j qleen. //
  \glb  neg neg com-e3-a3-traer-dir ninguna cosa //
  \glft `Ya no se trajo ninguna cosa.' //
\endgl
\xe


\ex
\begingl
  \gla R-eqaj-ch atz'yaq, x-\O-\O-k'am-ch chi-r-chooch, y x-\O-yuq chi-r-chooch ta' x-\O-\O-ij r-e r-ix\'oq-l, ni r-ech'elxiik, ni jun. //
  \glb e3-cargar-dir traje com-a3-e3-traer-dir pre-e3-casa and com-a3-venir pre-e3-casa neg com-a3-e3-decir e3-sr e3-mujer-sab ni e3-familiar ni uno //
  \glft `Se carg\'o el traje, se los trajo para su casa y vino a su casa no le dijo nada a su esposa, ni a sus familiares, a nadie.' //
\endgl
\xe

\ex
\begingl
  \gla Y j-pach t'-\O-el taq l xjooj x-\O-\O-ij taq re: "Y a-k\'uchj? Nin t-\O-a-b'an re?" //
  \glb and e3-compa\~nero inc-a3-salir pl pre baile com-a3-e3-decir pl dem y e2s-cuota int inc-a3-e2s-hacer dem  //
  \glft `Y sus compa\~neros del baile, le preguntaron: "?`Y tu cuota? ?`Qu\'e vas a hacer?"' //
\endgl
\xe

\ex
\begingl
  \gla "No, iin ta' t-\O-in-ye' in-k\'uchj." //
  \glb  no yo neg inc-a3-e1s-dar e1s-cuota //
  \glft `"No, yo no voy a dar mi cuota."' //
\endgl
\xe


\ex
\begingl
  \gla  Per nin t-\O-a-b'an r-e aw-itz'yaq? //
  \glb pero int inc-a3-e2s-hacer e3-sr a2s-traje //
  \glft ?`Pero qu\'e vas a hacer con tu traje? //
\endgl
\xe

\ex
\begingl
  \gla  Pues nin t-\O-in-b'an tre. //
  \glb pues int inc-a3-e1s-hacer dem //
  \glft `Pues qu\'e puedo hacer.' //
\endgl
\xe

\ex
\begingl
  \gla  T-\O-inw-ilb'e-j na ti-\O-taw q'iij y \{rik'u re'\} t-\O-iin-jut-un ch-a-xo'l-aq. //
  \glb inc-a3-e1s-esperar-sc part inc-b3-llegar d\'ia and despu\'es inc-a3-e1s-meter-ap pre-e2s-entre-pl  //
  \glft `Tengo que esperar que llegue el d\'ia y despu\'es me meto entre ustedes.' //
\endgl
\xe

\ex
\begingl
  \gla Jaan x-\O-che' taq re, "Per \{nin k'u\} t-\O-r-an wunaq li? Ta' ta' x-\O-ya' j-k\'uchj re r-itz'yaq." //
  \glb bueno com-a3-decir pl dem pero int inc-a3-e3-hacer hombre pre neg neg com-a3-dar e3-cuota dem e3-traje //
  \glft `Bueno le dijeron, "?`Pero qu\'e va ha hacer ese hombre? No dio su cuota para su traje."' //
\endgl
\xe

\ex
\begingl
  \gla "Y nin t-\O-qa-b'an re?" //
  \glb y int inc-a3-e1p-hacer dem //
  \glft `"?`Y qu\'e le vamos a hacer?"' //
\endgl
\xe

\ex
\begingl
  \gla  "Wa'x naq kla', t-\O-r-il li j-cholaj-l." //
  \glb estar cerca ah\'i inc-a3-e3-ver pre e3-forma-sab //
  \glft `"Que se quede ah\'i, que vea que hacer."' //
\endgl
\xe


\ex
\begingl
  \gla  X-\O-taw j-q'\'ijl roox q'iij r-e j-nimq'iij Simyel. //
  \glb com-b3-llegar e3-festividad tercera d\'ia e3-sr e3-fiesta \{San Miguel\} //
  \glft `Lleg\'o la festividad el tercer d\'ia de la fiesta de San Miguel.' //
\endgl
\xe

\ex
\begingl
  \gla  X-\O-joow taq y x-\O-koj-ch r-itz'yaq wunaq, T-\O-qop-qo-n! T-\O-qop-qo-n!, l q'aaq' r-itz'iiyaq. //   %rol>l>>l> rol<l><l>
  \glb com-a3-bailar pl y com-a3-poner-dir e3-traje hombre inc-a3-brillar-red-ap inc-a3-brillar-red-ap pre luz e3-traje //
  \glft `Bailaron y se puso su traje el hombre, !`Brillaba! !`Brillaba!, en la luz su traje.'  //
\endgl
\xe

\ex
\begingl
  \gla  Y x-\O-r-il taq juntir. //
  \glb  and com-a3-e3-ver pl todos //
  \glft `Y lo vieron todos.' //
\endgl
\xe

\ex
\begingl
  \gla "Lamas x-\O-a-me'-w aw-itz'yaq?" //
  \glb  int com-a3-e2s-recibir-enf e2s-traje //
  \glft `"?`D\'onde fuiste a traer tu traje?"' //
\endgl
\xe

\ex
\begingl
  \gla  "Ta' ke' a-puwaq, ta' a-meer, ta' a-kortiil." //
  \glb  neg adv e2s-dinero neg e2s-dinero neg e2s-riches  //
  \glft `"No ten\'ias dinero, no tenes riquezas."' //
\endgl
\xe



\ex
\begingl
  \gla  "At b'eyom ta'-n, y \{nin k'u\} man atz'yaq, li li x-\O-a-me'-w?"//
  \glb tu rico neg-enf y int dem traje pre pre com-a3-e2s-recibir-enf //
  \glft `"No ten\'ias nada y ese traje ?`d\'onde lo fuiste a traer?"' //
\endgl
\xe

\ex
\begingl
  \gla  Ta' t-\O-j-b'ij. //
  \glb neg inc-a3-e3-decir //
  \glft `No dec\'ia nada.' //
\endgl
\xe


\ex
\begingl
  \gla  Y ta' t-\O-\O-b'ij, t-\O-j-k'ot taq j-chii', "Per lamas x-\O-a-me'-w aw-itz'yaq?" //
  \glb  y neg inc-a3-e3-decir inc-a3-e2s-interrogate pl e3-sr pero int com-a3-e3-recibir-enf e2s-traje //
  \glft `Y no lo dec\'ia, lo interrogaban, "?`Pero d\'onde fuiste a traer tu traje?"' //
\endgl
\xe

\ex
\begingl
  \gla Ta' t-\O-\O-b'ij, va, t-\O-axan jun wi' jun ch-ij-xo'l-aq t-\O-\O-b'ij re: //
  \glb neg inc-a3e3--decir vaya inc-a3-pasar uno exs uno pre-3s-entre-pl inc-a3-e3-decir dem //
  \glft `No lo dec\'ia, va, vino uno que estaba entre ellos dijo:' //
\endgl
\xe

\ex
\begingl
  \gla  "\O-qa-ye' awryent re, \O-qa-q'ab'r-saj." //
  \glb  a3-e1s-dar licor dem a3-e1p-emborrachar-caus //
  \glft `"Demosle licor, emborrachemoslo."' //
\endgl
\xe

\ex
\begingl
  \gla  "Ya sach-e'l chaq j-b'aa, t-\O-j-yol-a' li ch-q-e lamas x-\O-ij-me'-w." //
  \glb  ya perder-pp part e3-cabeza inc-a3-e3-contar-sc pre pre-e1p-sr int com-e3-recibir-enf //
  \glft `"Ya inconsciente, nos ir\'a a contar donde lo fue a traer."' //
\endgl
\xe

\ex
\begingl
  \gla  "Jaan," x-\O-\O-che' taq, x-\O-\O-koj, x-\O-\O-ya' taq k\'um-b'l re. //
  \glb bueno com-a3-e3-decir pl com-a3-usar com-a3-dar pl medicine-inst dem  //
  \glft "Bueno," dijeron, le dieron licor." //
\endgl
\xe

\ex
\begingl
  \gla  X-\O-\O-q'ab'r-saj taq. //
  \glb com-a3-e3-emborrachar-caus pl  //
  \glft `Lo emborracharon.' //
\endgl
\xe


\ex
\begingl
  \gla  Ya laj q'ab'ar-k, x-\O-el-ch laj chii'. //
  \glb ya pre emborrachar-sv com-a3-salir-dir pre boca //
  \glft `Ya entre su borrachera, sali\'o de su boca.' //
\endgl
\xe

\ex
\begingl
  \gla  T-\O-ij-b'ij rechaq, "Pues lamas-tch x-\O-in-me' w-itz'yaq, jli b'a loom." //
  \glb inc-a3-e3-decir ellos pues int-dir com-a3-e1s-recibir e1s-traje all\'a pre cerro //
  \glft `Y les dijo, "Pues donde fui a traer mi traje, all\'a sobre el cerro."' //
\endgl
\xe

\ex
\begingl
  \gla  "Lamas loom?" //
  \glb int cerro //
  \glft `"?`Qu\'e cerro?"' //
\endgl
\xe

\ex
\begingl
  \gla  "Jli b'a Xoqoneb'." //
  \glb  all\'a pre Xoqoneb' //
  \glft `"All\'a en Xoqoneb'."' //
\endgl
\xe

\ex
\begingl
  \gla  "\{Nin k'u\} x-\O-a-b'an?" //
  \glb  int com-a3-e2s-hacer //
  \glft `"?`Qu\'e hiciste?"' //
\endgl
\xe

\ex
\begingl
  \gla "X-\O-el-ch jun nim-laj wunaq ch-w-iij." //
  \glb com-a3-salir-dir uno grande-sup hombre pre-e1s-sr //
  \glft `"Sali\'o un hombre grande detr\'as de mi."' //
\endgl
\xe


\ex
\begingl
  \gla  "Y x-\O-\O-ij ch-w-e que \{nen chak\} in-b'eson-k?" //
  \glb  y com-a3-e3-decir pre-e1s-sr que int e1-estar.triste-sc  //
  \glft `"Y me dijo que ?`por qu\'e estas triste?' //
\endgl
\xe

\ex
\begingl
  \gla  Cuando wi' laj q'aab' r-e r-e utziil r-e j-t'o'-w y kla' x-\O-in-me'-ch w-itz'yaq. //
  \glb  cuando exs pre mano e3-sr e3-sr bondad e3-sr e3-ayudar-sc y ah\'i com-a3-e1s-recibir-dir e1s-traje //
  \glft `Cuando ten\'ia entre sus manos alegr\'ia para ayudar y ah\'i fui a traer mi traje.' //
\endgl
\xe



\ex
\begingl
  \gla  K'ut-taj ch-q\'a-wch! //
  \glb ense\~nar-pas pre-e1p-rostro //
  \glft `!`Anda a ense\~narnos!' //
\endgl
\xe

\ex
\begingl
  \gla  Loq'ori ta' chki in-kwiin, ta' chki t-\O-in-te'. //
  \glb  ahorita neg neg e1s-poder neg neg inc-a3-e1s-encontrar //
  \glft `Ahorita ya no puedo, ya no lo encuentro.' //
\endgl
\xe

\ex
\begingl
  \gla  Ta' chki t-\O-in-te'. //
  \glb  neg neg inc-a3-e1s-encontrar //
  \glft `Ya no lo encuentro.' //
\endgl
\xe


\ex
\begingl
  \gla  Y kla' k'uli t-\O-j-yol taq qa-maam qa-tiit'. //
  \glb  y as\'i part inc-a3-e3-contar pl e1p-abuelo e1p-abuela //
  \glft `Y es as\'i como cuentan nuestros abuelos.' //
\endgl
\xe

\ex
\begingl
  \gla  Que kla' wi' jun meer j-kortiil qa-tinmit \{Tz'unun Kaab'\}. //
  \glb  que ah\'i exs uno dinero e3-riqueza e1p-pueblo Uspant\'an //
  \glft `Que ah\'i estaba la riqueza de nuestro pueblo Uspant\'an.' //
\endgl
\xe

\ex
\begingl
  \gla  \{Nin k'u chak\} x-\O-q'aje b'i-k? //
  \glb  int com-a3-regresar dir-sc  //
  \glft ?`Por qu\'e se regreso?' //
\endgl
\xe

\ex
\begingl
  \gla  \{Nin k'u chak\} x-\O-sach b'i-k? //
  \glb int com-a3-perder dir-sc //
  \glft `?`Por qu\'e se perdio?' //
\endgl
\xe

\ex
\begingl
  \gla  Mat x-\O-ij-yol na che'. //
  \glb neg com-a3-e3-contar adv rep //
  \glft `No lo hubiera contado dice.' //
\endgl
\xe


\ex
\begingl
  \gla  Mat x-\O-el na laj chii' wunaq, x-\O-ij-k'ol taj. //
  \glb neg com-a3-salir adv pre boca hombre com-a3-e3-guardar pl //
  \glft `No hubiera salido de la boca del hombre, lo hubiera guardado.' //
\endgl
\xe

\ex
\begingl
  \gla X-\O-ij-ch'uqt j-b'aa lo que x-\O-ij-saj re, j-puwaq qa-tinmit Tz'unun Kaab', \{ajwi' tna\} re. //
  \glb com-a3-e3-olvidar e3-cabeza lo que com-a3-e3-decir-caus dem e3-dinero e1p-pueblo Tz'unun Kaab' todav\'ia dem //
  \glft `No hubiera olvidado lo que le dijeron, el dinero de nuestro pueblo de Uspant\'an, a\'un estuviera.' //
\endgl
\xe

\ex
\begingl
  \gla  Cuand x-\O-r-etmaj wunaq, r-etmaj nim-laj wunaq. //
  \glb cuando com-a3-e3-saber hombre e3-saber grande-sup hombre //
  \glft `Cuando se enter\'o el hombre, se enter\'o el hombre grande.' //
\endgl
\xe

\ex
% \begingl
%   \gla  //
%   \glb //
%   \glft `Xyolsaaj jmeb'iil qatinmit, q'iij re xjyol.' //
% \endgl
\xe

\ex
\begingl
  \gla  X-\O-tormaj jun lel B'a Kameb', x-\O-nat-un b'i-k K\'ob'n. //
  \glb com-a3-abrir uno ventana B'a Kameb' com-a3-ver-ap dir-sc Cob\'an //
  \glft `Se abri\'o una ventana en B'a Kameb', viendo hacia Cob\'an.' //
\endgl
\xe


\ex
\begingl
  \gla X-\O-nat-un b'i-k Rab'inal, kla'. //
  \glb com-a3-ver-ap dir-sc Rab'inal ah\'i //
  \glft `Vio hacia Rabinal, ah\'i.' //
\endgl
\xe

\ex
\begingl
  \gla  X-\O-tok b'i j-b'ee j-meer j-kortiil. //
  \glb com-a3-buscar dir e3-camino e3-dinero e3-riqueza //
  \glft `Busc\'o su camino la riqueza.' //
\endgl
\xe

\ex
\begingl
  \gla  J'u-puwaq Tz'unun Kaab'. //
  \glb e3-diner Uspant\'an //
  \glft `El dinero de Uspant\'an.' //
\endgl
\xe

\ex
\begingl
  \gla  K'ixk'ol-il x-\O-tamaj \'ojr taq tziij. //
  \glb  dificultad-sab com-a3-encontrar antiguamente pl palabra //
  \glft `Dificultades se dieron en la antiguedad.' //
\endgl
\xe

\ex
\begingl
  \gla \'Ojr taq qa-maam qa-tiit' t-\O-b'eson taq. //
  \glb antiguamente pl e1p-abuelo e1p-abuela inc-a3-estar.triste pl //
  \glft `Nuestros abuelos quedaron tristes.' //
\endgl
\xe


\ex
\begingl
  \gla T-\O-aj-b'ij taq, kla' t-\O-awem \'ojr, nim-laj k'achee'laj. //
  \glb inc-a3-e3-decir pl ah\'i inc-b3-llegar antiguamente grande-sup bosque //
  \glft `Dec\'ian, que al llegar ah\'i antes, era una gran monta\~na.' //
\endgl
\xe

\ex
\begingl
  \gla Ta' qas t-\O-taw jun kla' porque k'an-i'n, k'an-i'n loq'-laj mundo kla'. //
  \glb neg part inc-b3-llegar uno ah\'i porque enojado-enf enojado-enf sagrado-sup mundo ah\'i //
  \glft `No llegaba uno ah\'i porque era enojado, enojado el sagrado mundo ah\'i.' //
\endgl
\xe

\ex
\begingl
  \gla  Ti li x-\O-kan yol-saaj j-wi'l qa-qaaj Jos\'e M\'endez. //
  \glb dem pre com-a3-dir contar-caus e3-sr e1p-padre Jos\'e M\'endez //
  \glft `Eso fue relatado por Don Jos\'e M\'endez.' //
\endgl
\xe

\ex
\begingl
  \gla  Ri li qa-maam qa-tiit' x-\O-kan yoloow taq y x-\O-\O-i'j taq, "X-\O-qa-k'ol tne." //
  \glb  dem pre e1p-abuelo e1p-abuelo com-a3-dir relatado pl y com-a3-e3-decir pl com-a3-e1p-guardar dem //
  \glft `Fueron nuestros abuelos quienes lo relataron y dijeron, "Lo hubieramos guardado."' //
\endgl
\xe

\ex
\begingl
  \gla  X-\O-qa-chajaj tne j-loq'\'ox-l y loq'-b'al puwaq r-e qa-tinmit Tz'unun Kaab'. //
  \glb com-a3-e1p-guidar dem e3-apreciar-sab y sagrado-inst diner e3-sr e1p-pueblo Tz'unun Kaab' //
  \glft `Hubieramos cuidado el sagrado dinero de nuestro pueblo Uspant\'an.' //
\endgl
\xe

\ex
\begingl
  \gla  Ri li t-\O-in-yol jun saqrab'em ri r-e. //
  \glb dem pre inc-a3-e1s-decir uno ma\~nana dem e3-sr //
  \glft `Esto es lo que yo relato en esta ma\~nana.' //
\endgl
\xe

\ex
\begingl
  \gla  R-ik'al junab' y t-\O-in-k'amwaj ch-aw-e ataq. //
  \glb  e3-sr a\~no y inc-a3-e1-agradecer pre-e2-sr ustedes //
  \glft `En este a\~no y les agradezco a ustedes.' //
\endgl
\xe

\end{document}